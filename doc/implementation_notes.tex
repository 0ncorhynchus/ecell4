\documentclass[english]{article}


\begin{document}

\title{Enhanced Green's Function Reaction Dynamics Implementation Notes}


\author{Koichi Takahashi}

\maketitle

\section*{Definitions}


\subsection*{Conventions}

\begin{itemize}
\item Units are in SI MKS unless otherwise noted.
\end{itemize}

\subsection*{Symbols}

\begin{description}
\item [{$D$}] a translational diffusion coefficient.
\item [{$a$}] a radius of the protective sphere.
\item [{$e$}] the base of natural logarithm.
\item [{$t$}] time.
\item [{$r$}] a distance from the origin.
\item [{$r_{0}$}] an initial distance of the particle from the origin.
\item [{$\mathbf{r}$}] a position in Cartesian coordinates.
\item [{$\mathbf{r_{0}}$}] an initial position in Cartesian coordinates
\item [{\textmd{$\vartheta_{i}$}}] Jacobi theta function as defined in
\emph{Abramowitz and Stegun}.
\item [{$\mathcal{U}$}] a unit uniform random variable.
\end{description}

\section{Green's functions}


\subsection{Free Single}

Cartesian form:

\begin{equation}
p(\mathbf{r},t)=
\frac{1}{(4\pi Dt)^{\frac{3}{2}}}e^{-\frac{|\mathbf{r}|^{2}}{4Dt}}.
\end{equation}


In spherical coordinates:

\begin{equation}
  p(r,t|r_{0},t_{0}=0)=
  \frac{1}{(4\pi Dt)^{\frac{3}{2}}}
  e^{-\frac{r^{2}+r_{0}^{2}-2rr_{0}\cos(\theta)}
    {4Dt}}.
\end{equation}


\subsubsection{Survival probability}

\begin{equation}
S(t)=1.
\end{equation}


\subsubsection{Sampling $r$}

\begin{equation}
\mathrm{erf}\left(\frac{r}{2 \sqrt{Dt}}\right) 
- \frac{e^{-\frac{r^2}{4 D t}} r}{\sqrt{\pi D t}}
\end{equation}


\subsubsection{Sampling $\theta$}


\subsection{Single with an absorbing sphere}

We find the following in section 14.7 of C\&J (eq. 17 in page 366),
that is for the sphere $0 \geq r>a$ with an absorbing surface and
an unit instantaneous spherical surface source at $r=r_{0},t=0$;

\begin{equation}
\frac{1}{2\pi arr_{0}}\sum_{n=1}^{\infty}e^{-\frac{Dn^{2}\pi^{2}t}{a^{2}}}\sin\frac{n\pi r}{a}\sin\frac{n\pi r_{0}}{a}.\end{equation}


The initial condition with a Dirac delta $\delta(r,t)$ corresponds to
taking the limit $r_{0}\rightarrow0$. Using 
\begin{equation}
  \lim_{r_{0}\rightarrow0}\left(\frac{a}{n\pi r_{0}}\sin\frac{n\pi
      r_{0}}{a}\right)=1,
\end{equation}
 we obtain the desired Green's function 
 \begin{equation}
   p_{1}(r,t)=\frac{1}{2a^{2}r}\sum_{n=1}^{\infty}n\, e^{-\frac{Dn^{2}\pi^{2}t}{a^{2}}}\sin\frac{n\pi r}{a}.
 \end{equation}


A conceptually identical equation can be found in eq (3.12) in \textit{Kalos
and Verlet 1974}. See also 9.3 (3) and (19) in C\&J.

Note the Jacobian in terms of $r$, $4\pi r^{2}$.


\subsubsection{Survival probability}

We want to calculate \begin{equation}
S(t)=\int_{0}^{a}4\pi r^{2}p_{1}(r,t)dr.\end{equation}


We get \begin{equation}
S(t)=-2\sum_{n=1}^{\infty}(-1)^{n}e^{-\frac{Dn^{2}\pi^{2}t}{a^{2}}}.\end{equation}


Numerically, this series is not a good behaving one and does not converge
well. Using the fact that \begin{equation}
\sum_{n=1}^{\infty}(-1)^{(n-1)}e^{-n^{2}w}=\frac{1}{2}(1-\vartheta_{4}(0,e^{-w})),\end{equation}
 where $\vartheta$ is Jacobi's theta, we rewrite $S(t)$ as \begin{equation}
S(t)=1-\vartheta_{4}(0,e^{-\frac{D\pi^{2}t}{a^{2}}}).\end{equation}
 Then we can efficiently evaluate this function through a product
representation \begin{equation}
\vartheta_{4}(0,q)=\prod_{n=1}^{\infty}(1-q^{2n})(1-q^{(2n-1)})^{2}.\end{equation}



\subsubsection{Sampling $r$}

Unless we are somehow able to obtain an inverse function, one simple
but not awfully efficient way to sample $r$ is to use another integral
of $p_{1}(r,t)$, \begin{equation}
\int_{0}^{r'}4\pi r^{2}p_{1}(r,t)dr=\frac{2}{a\pi}\sum_{n=1}^{\infty}\frac{e^{-\frac{Dn^{2}\pi^{2}t}{a^{2}}}\left(a\sin\frac{n\pi r'}{a}-n\pi r'\cos\frac{n\pi r'}{a}\right)}{n}.\end{equation}


Using this, and eq. (6), with nonlinear programming, we find $r'$
so that 
\begin{equation}
   \mathcal{U}-\frac{\int_{0}^{r'}4\pi r^{2}p_{1}(r,t)dr}{S(t)}=0,
\end{equation}
 where $\mathcal{U}$ is a unit uniform random number.


\section{Main algorithm}

\subsection{Classes}

\subsubsection{Species}

\paragraph{Properties}

\begin{itemize}
\item[radius] the radius of a particle of this Species in [$m$].
\item[D] the diffusion coefficient of particles of this Species in [$m^2/s$].
\end{itemize}

\subsubsection{Particle}

\paragraph{Properties}

\begin{itemize}
\item[pos] the position of this Particle in [$m$].
\item[species] the Species this Particle belongs to.
\end{itemize}

\subsubsection{Single}

\paragraph{Properties}

\begin{itemize}
\item[eventType] the time of the next event, either REACTION or ESCAPE.
\item[dt] the time to the next event from the lastTime.
\item[lastTime] the last time this Single was updated.
\item[mobilityRadius] $shellSize - particle.radius$.  This
  indicates the maximum displacement the particle can make within the
  shell.
\item[particle] the particle that this Single represents.
\item[pos] the position of this Single defined as the position of the
  particle.
\item[shellSize] the shell size of this Single.


\end{itemize}

\paragraph{Methods}

\begin{itemize}
\item[burst( t )] This method updates the position of the particle at
  time $t$.
  \begin{enumerate}
  \item[precondition] $lastTime <= t <= lastTime + dt$.

    \item draw $r$ from the Greens Function (which one?).

    \item propagate( r, t )
  \end{enumerate}

\item[determineNextEvent()] This method updates dt and eventType of
  this Single.

%\item[displace( r )] This method displaces the particle to a random
%  point on the spherical surface of distance r from the original
%  position.

\item[propagate( r, t )] This method displaces the particle to a
  random point on the spherical surface of distance r from the
  original position, set lastTime to t, then reset the shell size to
  the radius of the particle, dt to zero, and eventType to ESCAPE.

\end{itemize}

\subsubsection{Pair}

\paragraph{Properties}

\begin{itemize}
\item[D] $single1.particle.D + single2.particle.D$


\item[CoM] Center-of-Mass  
  $= ( \frac{\sqrt{D2}}{\sqrt{D1}} * single1.pos +
  \frac{\sqrt{D1}}{\sqrt{D2}} * single2.pos ) / (
  \frac{\sqrt{D2}}{\sqrt{D1}}+\frac{\sqrt{D1}}{\sqrt{D2}})$

\item[single1, single2] the pair of singles this Pair is responsible for.


\item[eventType] the type of the next event, one of REACTION,
  ESCAPE1, or ESCAPE2.

\item[lastTime] the last time this Pair was updated.

\item[dt] the time to the next event from the lastTime.

\item[shellSize] the shell size of this Pair.

\end{itemize}

\paragraph{Methods}

\begin{itemize}
\item 
\end{itemize}


\subsection{Handling the Squeezing conditions}

When two particles are very close to each other, a Pair must be formed
to advance time.  However, the distance from the center-of-mass of the
Pair to the closest neighbor of this pair of particles is very close
to or is less than the minimum mobility radius of the pair, the Pair
can no longer hold, and the couple of particles is said to be 'squeezed'
by the closest neighbor of the Pair.

(FIXME: a figure here)

There are three classes of methods in avoiding and handling the squeezing
conditions.

\begin{enumerate}
\item To avoid (or minimize the possibility of) squeezing in the first place.

\item To recover from a squeezing condition by displacing one or more
  particles in some arbitrary methods.

\item To allow squeezing to happen under some controlled conditions,
  watch influenced objects, and, when necessary, intervene the
  physical simulation in order to protect consistency of the simulator
  until the squeezing condition is dissolved.
\end{enumerate}


few different ways to handle squeezing conditions.  One
possible scheme is described in the following.






\end{document}
