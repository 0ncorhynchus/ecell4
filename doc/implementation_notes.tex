\documentclass[english]{article}


\begin{document}

\title{Enhanced Green's Function Reaction Dynamics Implementation Notes}


\author{Koichi Takahashi}

\maketitle

\section*{Definitions}


\subsection*{Conventions}

\begin{itemize}
\item Units are in SI MKS unless otherwise noted.
\end{itemize}

\subsection*{Symbols}

\begin{description}
\item [{$D$}] a translational diffusion coefficient.
\item [{$a$}] a radius of the protective sphere.
\item [{$e$}] the base of natural logarithm.
\item [{$t$}] time.
\item [{$r$}] a distance from the origin.
\item [{$r_{0}$}] an initial distance of the particle from the origin.
\item [{$\mathbf{r}$}] a position in Cartesian coordinates.
\item [{$\mathbf{r_{0}}$}] an initial position in Cartesian coordinates
\item [{\textmd{$\vartheta_{i}$}}] Jacobi theta function as defined in
\emph{Abramowitz and Stegun}.
\item [{$\mathcal{U}$}] a unit uniform random variable.
\end{description}

\subsection*{Terminology}

\begin{description}
\item[Distance (between objects)] Distance between positions of two
  objects in space.
\item[Gap (between objects)] Smallest distance between surfaces of two objects.
\item[Protective Sphere / Domain / Shell] Excluive region for an object.
\item[Mobility radius (of an object)] The maximum displacement an
  object can make within its protective sphere.

\end{description}



\section{Green's functions}


\subsection{Free Single}

Cartesian form:

\begin{equation}
p(\mathbf{r},t)=
\frac{1}{(4\pi Dt)^{\frac{3}{2}}}e^{-\frac{|\mathbf{r}|^{2}}{4Dt}}.
\end{equation}


In spherical coordinates:

\begin{equation}
  p(r,t|r_{0},t_{0}=0)=
  \frac{1}{(4\pi Dt)^{\frac{3}{2}}}
  e^{-\frac{r^{2}+r_{0}^{2}-2rr_{0}\cos(\theta)}
    {4Dt}}.
\end{equation}


\subsubsection{Survival probability}

\begin{equation}
S(t)=1.
\end{equation}


\subsubsection{Sampling $r$}

\begin{equation}
\mathrm{erf}\left(\frac{r}{2 \sqrt{Dt}}\right) 
- \frac{e^{-\frac{r^2}{4 D t}} r}{\sqrt{\pi D t}}
\end{equation}


\subsubsection{Sampling $\theta$}


\subsection{Single with an absorbing sphere}

We find the following in section 14.7 of C\&J (eq. 17 in page 366),
that is for the sphere $0 \geq r>a$ with an absorbing surface and
an unit instantaneous spherical surface source at $r=r_{0},t=0$;

\begin{equation}
\frac{1}{2\pi arr_{0}}\sum_{n=1}^{\infty}e^{-\frac{Dn^{2}\pi^{2}t}{a^{2}}}\sin\frac{n\pi r}{a}\sin\frac{n\pi r_{0}}{a}.\end{equation}


The initial condition with a Dirac delta $\delta(r,t)$ corresponds to
taking the limit $r_{0}\rightarrow0$. Using 
\begin{equation}
  \lim_{r_{0}\rightarrow0}\left(\frac{a}{n\pi r_{0}}\sin\frac{n\pi
      r_{0}}{a}\right)=1,
\end{equation}
 we obtain the desired Green's function 
 \begin{equation}
   p_{1}(r,t)=\frac{1}{2a^{2}r}\sum_{n=1}^{\infty}n\, e^{-\frac{Dn^{2}\pi^{2}t}{a^{2}}}\sin\frac{n\pi r}{a}.
 \end{equation}


A conceptually identical equation can be found in eq (3.12) in \textit{Kalos
and Verlet 1974}. See also 9.3 (3) and (19) in C\&J.

Note the Jacobian in terms of $r$, $4\pi r^{2}$.


\subsubsection{Survival probability}

We want to calculate \begin{equation}
S(t)=\int_{0}^{a}4\pi r^{2}p_{1}(r,t)dr.\end{equation}


We get \begin{equation}
S(t)=-2\sum_{n=1}^{\infty}(-1)^{n}e^{-\frac{Dn^{2}\pi^{2}t}{a^{2}}}.\end{equation}


Numerically, this series is not a good behaving one and does not converge
well. Using the fact that \begin{equation}
\sum_{n=1}^{\infty}(-1)^{(n-1)}e^{-n^{2}w}=\frac{1}{2}(1-\vartheta_{4}(0,e^{-w})),\end{equation}
 where $\vartheta$ is Jacobi's theta, we rewrite $S(t)$ as \begin{equation}
S(t)=1-\vartheta_{4}(0,e^{-\frac{D\pi^{2}t}{a^{2}}}).\end{equation}
 Then we can efficiently evaluate this function through a product
representation \begin{equation}
\vartheta_{4}(0,q)=\prod_{n=1}^{\infty}(1-q^{2n})(1-q^{(2n-1)})^{2}.\end{equation}



\subsubsection{Sampling $r$}

Unless we are somehow able to obtain an inverse function, one simple
but not awfully efficient way to sample $r$ is to use another integral
of $p_{1}(r,t)$, \begin{equation}
\int_{0}^{r'}4\pi r^{2}p_{1}(r,t)dr=\frac{2}{a\pi}\sum_{n=1}^{\infty}\frac{e^{-\frac{Dn^{2}\pi^{2}t}{a^{2}}}\left(a\sin\frac{n\pi r'}{a}-n\pi r'\cos\frac{n\pi r'}{a}\right)}{n}.\end{equation}


Using this, and eq. (6), with nonlinear programming, we find $r'$
so that 
\begin{equation}
   \mathcal{U}-\frac{\int_{0}^{r'}4\pi r^{2}p_{1}(r,t)dr}{S(t)}=0,
\end{equation}
 where $\mathcal{U}$ is a unit uniform random number.


\section{Main algorithm}

The main part of the implementation of EGFRD algorithm is described
in this section.  Here I avoided details and concentrated on basic
concepts, flows and schemes that should be useful to understand
the implementation.  

\subsection{Classes}

\subsubsection{Species}

\paragraph{Properties}

\begin{itemize}
\item[radius] the radius of a particle of this Species in [$m$].
\item[D] the diffusion coefficient of particles of this Species in [$m^2/s$].
\end{itemize}

\subsubsection{Particle}

\paragraph{Properties}

\begin{itemize}
\item[pos] the position of this Particle.
\item[species] the Species this Particle belongs to.
\end{itemize}

\subsubsection{Single}

\paragraph{Properties}

\begin{itemize}
\item[eventType] the time of the next event, either SINGLE_REACTION or 
  SINGLE_ESCAPE.
\item[dt] the time to the next event from the lastTime.
\item[lastTime] the last time this Single was updated.
\item[mobilityRadius] $shellSize - particle.radius$.  This
  indicates the maximum displacement the particle can make within the
  shell.
\item[particle] the particle that this Single represents.
\item[pos] the position of this Single defined as the position of the
  particle.
\item[shellSize] the shell size of this Single.


\end{itemize}

\paragraph{Methods}

\begin{itemize}
\item[burst( t )] This method updates the position of the particle at
  time $t$.
  \begin{enumerate}
  \item[precondition] $lastTime <= t <= lastTime + dt$.

    \item draw $r$ from the Greens Function (which one?).

    \item propagate( r, t )
  \end{enumerate}

\item[determineNextEvent()] This method updates dt and eventType of
  this Single.

%\item[displace( r )] This method displaces the particle to a random
%  point on the spherical surface of distance r from the original
%  position.

\item[propagate( r, t )] This method displaces the particle to a
  random point on the spherical surface of distance r from the
  original position, set lastTime to t, then reset the shell size to
  the radius of the particle, dt to zero, and eventType to SINGLE_ESCAPE.

\end{itemize}

\subsubsection{Pair}

\paragraph{Properties}

\begin{itemize}
\item[CoM] Center-of-Mass  
  $= ( \frac{\sqrt{D2}}{\sqrt{D1}} * single1.pos +
  \frac{\sqrt{D1}}{\sqrt{D2}} * single2.pos ) / (
  \frac{\sqrt{D2}}{\sqrt{D1}}+\frac{\sqrt{D1}}{\sqrt{D2}})$

\item[D_tot] Total of the diffusion constants of the pair of particles.
 $single1.particle.D + single2.particle.D$

\item[D_geom] Geometric mean of the diffusion constants of the pair of
  particles.  $\sqrt{ single1.particle.D \cdot single2.particle.D}$.
  This defines the diffusion constant of the CoM $D_{CoM}$


\item[single1, single2] the pair of singles this Pair is responsible for.

\item[eventType] the type of the next event, one of SINGLE_REACTION,
  PAIR_REACTION, IV_ESCAPE, or COM_ESCAPE.

\item[lastTime] the last time this Pair was updated.

\item[dt] the time to the next event from the lastTime.

\item[shellSize] the shell size of this Pair.

\end{itemize}

\paragraph{Methods}

\begin{itemize}
\item 
\end{itemize}


\subsection{Pair implementation}

\subsubsection{Determining Next Event}

\paragraph{Determining $a_r$ and $a_R$}

Here we have some degrees of freedom in Pair parameters, such as how
to determine the ratio between $a_r$ and $a_R$ given certain size of
the shell.  The objective here is to maximize the probability for the
pair to react, not to escape either through the protective shells of
the CoM or the inter-particle vector.  If the pair is non-reactive, on
the other hand, we want to maximize the time to escape by setting
these two parameters.  To do a perfect job here can be as hard as 
solving some equations involving the Green's functions analytically.
However, the empirical strategy described below works reasonably
good in most cases.

We want to equalize the expected mean time for the CoM to escape
through $a_R$ and for the inter-particle vector to escape through
$a_r$.  To keep the problem handy, here we ignore the boundary
at $\sigma$.  Then, we can expect that the mean time for the CoM
and the inter-particle vector to arrive at the absorbing shell
are

\begin{eqnarray}
  \Delta t_r &=& \frac{a^{\{A,B\}}_r - r^{\{A,B\}}_0 - \sigma^{\{A,B\}}} 
  {\sqrt{D_{tot}}},\\
  \Delta t_R &=& \frac{a_R}{\sqrt{D_{CoM}}},
\end{eqnarray}
respectively, where ${\{A,B\}}$ is either $A$ or $B$, depending on which
one of the particles define the protective shell.


Use the following to set $a_R$ and $a_r$

\begin{eqnarray}
  a_R &=& \frac{\sqrt{D_{CoM}} 
    (D_A ( a - r_0 - \sigma_A ) + D_B (a - \sigma_A))}
  {(D_A {D_B}^5)^{1/4} + D_A( \sqrt{D_{CoM}} + \sqrt{D_{tot}})}, \\
  a_r &=& \frac{D_{tot} 
    (\sqrt{D_{CoM}} r_0 + \sqrt{D_{tot}} (a - \sigma_A))}
  {(D_A {D_B}^5)^{1/4} + D_A( \sqrt{D_{CoM}} + \sqrt{D_{tot}})},
\end{eqnarray}
when
\begin{math}
  ( D_A = D_B \wedge \sigma_A > \sigma_B ) \ \vee \ \\
  \left ( ( D_A > D_B ) \wedge a_r > a_{thr} \right ) \ \vee \
  \left ( ( D_A < D_B ) \wedge a_r < a_{thr} \right ), \\
  a_{thr} = \frac{( D_A + D_B ) (\sigma_B - \sigma_A)}{D_A - D_B},
\end{math}
that means the particle $A$ defines the Pair shell.  Otherwise,
\begin{eqnarray}
  a_R &=& \frac{\sqrt{D_{CoM}} 
    (D_B ( a - r_0 - \sigma_B ) + D_A (a - \sigma_B))}
  {({D_A}^5 D_B)^{1/4} + D_B( \sqrt{D_{CoM}} + \sqrt{D_{tot}})}, \\
  a_r &=& \frac{D_{tot} 
    (\sqrt{D_{CoM}} r_0 + \sqrt{D_{tot}} (a - \sigma_B))}
  {({D_A}^5 D_B)^{1/4} + D_B( \sqrt{D_{CoM}} + \sqrt{D_{tot}})},
\end{eqnarray}

When $D_A = D_B \wedge \sigma_A = \sigma_B$, the choice of shell-defining particle
doesn't make any difference.








\subsection{Handling the Squeezing Conditions}

When two particles are very close to each other, a Pair must be formed
to advance time.  However, the distance from the center-of-mass of the
Pair to the closest neighbor of this pair of particles is very close
to or is less than the minimum mobility radius of the pair, the Pair
can no longer hold, and the couple of particles is said to be 'squeezed'
by the closest neighbor of the Pair.

(FIXME: a figure here)

There are three types of methods in avoiding and handling the squeezing
conditions.

\begin{enumerate}
\item To avoid (or minimize the possibility of) squeezing in the first place.

\item To recover from a squeezing condition by displacing one or more
  particles in some arbitrary methods.

\item To allow squeezing to happen under some controlled conditions,
  watch influenced objects, and, when necessary, intervene the
  physical simulation in order to protect consistency of the simulator
  until the squeezing condition is dissolved.
\end{enumerate}

It is unclear if methods of type (1) that can avoid squeezing to
happen completely can be possible, thus this type of method shall be used in
conjunction with either (2) 'recovery' methods or (3) 'allow and
watch' methods.

Here I describe a type of 'allow and watch' methods that is included
in the implementation.



\subsection{Subvolumes and Cyclic Boundaries}





\end{document}
