\title{New p1 for GFRD using first-passage process}
\author{Koichi Takahashi}
\date{}
\documentclass{article}
\begin{document}
\maketitle
\begin{center}
Initial version on Dec 18, 2006.\\
Revised on Dec 19, 2006.
\end{center}

\section{The Green's function}

We find the following in section 14.7 of C\&J (eq. 17 in page 366),
that is for the sphere $0 \geq r > a$ with an absorbing surface
and an unit instantaneous spherical surface source at $r = r_0, t = 0$;

\begin{equation}
v = \frac{1}{2 \pi a r r_0} \sum_{n=1}^{\infty} e^{- \frac{D n^2 \pi^2 t}{a^2} }
\sin \frac{n \pi r}{a} \sin \frac{n \pi r_0}{a}.
\end{equation}

The initial condition with a Dirac delta $\delta(r,t)$ corresponds
to taking the limit $r_0 \rightarrow 0$.  Using
\begin{equation}
\lim_{r_0 \rightarrow 0} 
\left( \frac{a}{n \pi r_0} \sin \frac{n \pi r_0}{a}\right) = 1,
\end{equation}
we obtain the desired Green's function
\begin{equation}
p_1(r,t) = 
\frac{1}{2 a^2 r} \sum_{n=1}^{\infty} n \, e^{- \frac{D n^2 \pi^2 t}{a^2} }
\sin \frac{n \pi r}{a}.
\end{equation}

A conceptually identical equation can be found in eq (3.12) in
{\it Kalos and Verlet 1974}.    See also 9.3 (3) and (19) in C\&J.

The Jacobian in terms of $r$ is $4 \pi r^2$.

\section{Survival probability}

We want to calculate
\begin{equation}
S(t) = \int_{0}^{a} 4 \pi r^2 p_1(r,t) dr.
\end{equation}

We get
\begin{equation}
S(t) = -2 \sum_{n=1}^{\infty} (-1)^n e^{- \frac{D n^2 \pi^2 t}{a^2}}.
\end{equation}

This series is not a good behaving one and does not converge well.
Using the fact that
\begin{equation}
\sum_{n=1}^{\infty} (-1)^{(n-1)} e^{- n^2 w} = 
\frac{1}{2} ( 1 - \vartheta_4( 0, e^{-w} ) ),
\end{equation}
where $\vartheta$ is Jacobi theta, we rewrite $S(t)$ as 
\begin{equation}
S(t) = 1 - \vartheta_4( 0, e^{- \frac{D \pi^2 t}{a^2}} ).
\end{equation}
Then we can efficiently evaluate this function with a product 
representation
\begin{equation}
\vartheta_4( 0, q ) = \prod_{n=1}^{\infty} (1-q^{2n})(1-q^{(2n-1)})^2.
\end{equation}

\section{Sampling $r$}

Unless we are somehow able to obtain an inverse function, one
simple but not awfully efficient way to sample $r$ is to use another
integral of $p_1(r,t)$,
\begin{equation}
\int_{0}^{r'} 4 \pi r^2 p_1(r,t) dr = \frac{2}{a \pi} 
\sum_{n=1}^{\infty} \frac{e^{- \frac{D n^2 \pi^2 t}{a^2}}
\left( a \sin \frac{n \pi r'}{a} - n \pi r' \cos \frac{n \pi r'}{a} \right)}
{n}.
\end{equation}

Using this, and eq. (6), with nonlinear programming, we find $r'$ so that
\begin{equation}
\xi - \frac{\int_{0}^{r'} 4 \pi r^2 p_1(r,t) dr}{S(t)} = 0,
\end{equation}
where $\xi$ is a unit uniform random number.

\end{document}